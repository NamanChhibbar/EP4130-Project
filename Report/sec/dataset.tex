\section{Dataset}

We choose to use the full catalog of the \href{https://data.galaxyzoo.org/}{\textcolor{blue}{Galaxy Zoo 1}} dataset as our input. This dataset provides us with an object ID, coordinates to where a celestial object is, and a one-hot encoding of the category of the galaxy (Elliptical, Spiral) as well as an attribute that reflects the quality of the classification. The \href{https://classic.sdss.org/dr7/}{\textcolor{blue}{Sloan Digital Sky Survey}} provides an API from which we can fetch images of galaxies given their coordinates. Out of the 600K+ images in the original dataset, we picked only about 10K+ high-quality data points. The API provides functionality to specify size of the image. We used this API to get a 1000 images each of Elliptical and Spiral galaxies. Based on initial analysis, we believed that 512x512 is a reasonable input size. However, at the time of applying PCA to this dataset, we had to reduce the dimensions further to 128x128 as the RAM was not sufficient while running PCA on the dataset.

Additionally, we web-scraped about 200 images of Irregular galaxies; removed duplicate (faulty) images and spiral or elliptical images from them manually. Later, data augmentation is applied on them to get about 1000 images of Irregular category. We also scraped 828 non-celestial object images and added them to our dataset and labelled them as Invalid. Finally our generated dataset consists of 991 elliptical, 1001 spiral, 1000 irregular galaxies and 828 invalid images adding up to a total of 3820 images. Each image is to be classified in one of the following four categories: \textit{Elliptical, Spiral, Irregular, and Invalid}.
