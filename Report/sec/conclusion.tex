\section{Conclusions}

Through our experiments, we have explored various machine learning and deep learning models for automating the identification of galaxies based on image data. Our findings demonstrate that the CNN model achieved the highest performance on our dataset consisting of elliptical, spiral, irregular, and invalid images, achieving a classification accuracy of \(97\%\). While the CNN model showed superior performance, there is potential to further enhance accuracy by leveraging ensemble methods from traditional machine learning classifiers.

We believe that the techniques developed for galaxy image classification can be extended to other astronomical data types, such as nebulae, star clusters, or images containing multiple astronomical objects. The models we have created hold practical value for various algorithms in astronomy.

In conclusion, we introduce a new benchmark for galaxy image classification and highlight our best-performing model, the CNN, achieving a remarkable \(97\%\) classification accuracy, surpassing previous attempts by at least \(5\%\).

During this project, we gained valuable experience in various machine learning tasks, including:

\begin{itemize}
    \item Web-scraping images using SDSS APIs and astropy
    \item Image data augmentation with imgaug
    \item Applying PCA on images using scikit-learn
    \item Implementing Random Forests and SVMs using scikit-learn
    \item Building MLP and CNN models using Keras
    \item Generating classification reports and ROC curves using matplotlib and scikit-learn
    \item Visualizing results with matplotlib
\end{itemize}